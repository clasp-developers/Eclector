\chapter{Recovering from errors}

\section{Recovering from errors}
\label{sec:recovering-from-errors}

\sysname{} offers extensive support for recovering from many syntax
errors, continuing to read from the input stream and return a result
that somewhat resembles what would have been returned in case the
syntax had been valid.  To this end, a restart named
\texttt{eclector.reader:recover} is established when recoverable
errors are signaled.  Like the standard \commonlisp{} restart
\texttt{cl:continue}, this restart can be invoked by a function of the
same name:

\Defun {recover} {\optional condition}

This function recovers from an error by invoking the most recently
established applicable restart named \texttt{eclector.reader:recover}.
If no such restart is currently established, it returns \texttt{nil}.
If \textit{condition} is non-\texttt{nil}, only restarts that are
either explicitly associated with \textit{condition}, or not
associated with any condition are considered.

When a \texttt{read} call during which error recovery has been
performed returns, \sysname{} tries to return an object that is
similar in terms of type, numeric value, sequence length, etc. to what
would have been returned in case the input had been well-formed.  For
example, recovering after encountering the invalid digit in
\texttt{\#b11311} returns either the number \texttt{\#b11011} or the
number \texttt{\#b11111}.

\section{Recoverable errors}
\label{sec:recoverable-errors}

A syntax error and a corresponding recovery strategy are characterized
by the type of the signaled condition and the report of the
established \texttt{eclector.reader:recover} restart respectively.
Attempting to list and describe all examples of both would provide
little insight.  Instead, this section describes different classes of
errors and corresponding recovery strategies in broad terms:

\newcommand{\RecoverExample}[2]{\texttt{#1} $\rightarrow$ \texttt{#2}}

\begin{itemize}
\item Replace a missing numeric macro parameter or ignore an invalid
  numeric macro parameter.  Examples: \RecoverExample{\#=1}{1},
  \RecoverExample{\#5P"."}{\#P"."}

\item Add a missing closing delimiter.  Examples:
  \RecoverExample{"foo}{"foo"}, \RecoverExample{(1 2}{(1 2)},
  \RecoverExample{\#(1 2}{\#(1 2)}, \RecoverExample{\#C(1 2}{\#C(1 2)}

\item Replace an invalid digit or an invalid number with a valid one.
  This includes digits which are invalid for a given base but also
  things like $0$ denominator.  Examples: \RecoverExample{\#12rc}{1},
  \RecoverExample{1/0}{1}, \RecoverExample{\#C(1 :foo)}{\#C(1 1)}

\item Replace an invalid character with a valid one.  Example:
  \RecoverExample{\#\textbackslash{}foo}{\#\textbackslash{}?}

\item Invalid constructs can sometimes be ignored.  Examples:
  \RecoverExample{(,1)}{(1)}, \RecoverExample{\#S(foo :bar 1 2
    3)}{\#S(foo :bar 1)}

\item Excess parts can often be ignored.  Examples:
  \RecoverExample{\#C(1 2 3)}{\#C(1 2)}, \RecoverExample{\#2(1 2
    3)}{\#2(1 2)}

\item Replace an entire construct by some fallback value.  Example:
  \RecoverExample{\#S(5)}{nil}, \RecoverExample{(\#1=)}{(nil)}

\end{itemize}

\section{Potential problems}
\label{sec:potential-problems}

Note that attempting to recover from syntax errors may lead to
apparent success in the sense that the \texttt{read} call returns an
object, but this object may not be what the caller wanted.  For
example, recovering from the missing closing \texttt{"} in the
following example

\begin{Verbatim}[frame=single]
(defun foo (x y)
  "My documentation string
  (+ x y))
\end{Verbatim}

results in \verb!(DEFUN FOO (X Y) "My documentation string<newline>  (+ x y))")!,
not \verb!(DEFUN FOO (X Y) "My documentation string" (+ x y))!.
