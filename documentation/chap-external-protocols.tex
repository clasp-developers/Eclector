\chapter{External protocols}

\section{Packages}

\subsection{Package for ordinary reader features}
\label{sec:package-ordinary-reader-features}

The package for ordinary reader features is named
\texttt{eclector.reader}.  To use features of this package, we
recommend the use of explicit package prefixes, simply because this
package shadows and exports names that are also exported from the
\texttt{common-lisp} package.  Importing this package will likely
cause conflicts with the \texttt{common-lisp} package otherwise.

\subsection{Package for readtable features}
\label{sec:package-readtable-features}

The package for readtable-related features is named
\texttt{eclector.readtable}.  To use features of this package, we
recommend the use of explicit package prefixes, simply because this
package shadows and exports names that are also exported from the
\texttt{common-lisp} package.  Importing this package will likely
cause conflicts with the \texttt{common-lisp} package otherwise.

\subsection{Package for parse result construction features}
\label{sec:package-parse-result}

The package for features related to the creation of client-defined
parse results is named \texttt{eclector.parse-result}.  Although this
package does not shadow any symbol in the \texttt{common-lisp}
package, we still recommend the use of explicit package prefixes to
refer to symbols in this package.

\subsection{Package for CST features}
\label{sec:package-cst-features}

The package for features related to the creation of concrete syntax
trees is named \texttt{eclector.concrete-syntax-tree}.  Although this
package does not shadow any symbol in the \texttt{common-lisp}
package, we still recommend the use of explicit package prefixes to
refer to symbols in this package.

\section{Ordinary reader features}

In this section, symbols written without package marker are in the
\texttt{eclector.reader} package
\seesec{sec:package-ordinary-reader-features}

\Defun {read} {\optional (input-stream \texttt{*standard-input*})\\
  (eof-error-p \texttt{t})
  (eof-value \texttt{nil})
  (recursive-p \texttt{nil})}

This function is the main entry point for the ordinary reader.  It is
entirely compatible with the standard \commonlisp{} function with the
same name.

\Defvar {*client*}

This variable is used by several generic functions called by
\texttt{read}.  The default value of the variable is \texttt{nil}.
Client code that wants to override or extend the default behavior of
some generic function of \sysname{} should bind this variable to some
standard object and provide a method on that generic function,
specialized to the class of that standard object.

\Defgeneric {read-common} {client input-stream eof-error-p eof-value}

This generic function is called by \texttt{read}, passing it the value
of the variable \texttt{*client*} and the corresponding parameters.
Client code can add methods on this function, specializing them to the
client class of its choice.  The actions that \texttt{read} need to
take for different values of the parameter \texttt{recursive-p} have
already been taken before \texttt{read} calls this generic function.

\Defgeneric {note-skipped-input} {client input-stream reason}

This generic function is called whenever the reader skips some input
such as a comment or a form that must be skipped because of a reader
conditional.  It is called with the value of the variable
\texttt{*client*}, the input stream from which the input is being read
and an object indicating the reason for skipping the input.  The
default method on this generic function does nothing.  Client code can
supply a method that specializes to the client class of its choice.

When this function is called, the stream is positioned immediately
\emph{after} the skipped input.  Client code that wants to know both
the beginning and the end of the skipped input must remember the
stream position before the call to \texttt{read} was made as well as
the stream position when the call to this function is made.

\Defvar {*skip-reason*}

This variable is used by the reader to determine why a range of input
characters has been skipped.  To this end, internal functions of the
reader as well as reader macros can set this variable to a suitable
value before skipping over some input.  Then, after the input has been
skipped, the generic function \texttt{note-skipped-input} is called
with the value of the variable as its \textit{reason} argument.

As an example, the method on \texttt{note-skipped-input} specialized
to\\
\texttt{eclector.parse-result:parse-result-client} relays the reason
and position information to the client by calling the\\
\texttt{eclector.parse-result:make-skipped-input-result} generic
function~\seesec{sec:parse-result-construction-features}

\Defgeneric {read-token} {client input-stream eof-error-p eof-value}

This generic function is called by \texttt{read-common} when it has
been detected that a token should be read.  This function is
responsible for accumulating the characters of the token and then
calling \texttt{interpret-token} (see below) in order to create and
return a token.

\Defgeneric {interpret-token} {client input-stream token escape-ranges}

This generic function is called by \texttt{read-token} in order to
create a token from accumulated token characters.  The parameter
\textit{token} is a string containing the characters that make up the
token.  The parameter \textit{escape-ranges} indicates ranges of
characters read from \textit{input-stream} and preceded by a character
with single-escape syntax or delimited by characters with
multiple-escape syntax.  Values of \textit{escape-ranges} are lists of
elements of the form \texttt{(\textit{start}\ .\ \textit{end})} where
\texttt{\textit{start}} is the index of the first escaped character
and \texttt{\textit{end}} is the index \textit{following} the last
escaped character.  Note that \texttt{\textit{start}} and
\texttt{\textit{end}} can be identical indicating no escaped
characters.  This can happen in cases like \texttt{a||b}.  The
information conveyed by the \textit{escape-ranges} parameter is used
to convert the characters in \textit{token} according to the
\emph{readtable case} of the current readtable before a token is
constructed.

\Defgeneric {interpret-symbol-token} {client input-stream token \\
  position-package-marker-1 position-package-marker-2}

This generic function is called by the default method on
\texttt{interpret-token} when the syntax of the token corresponds to
that of a valid symbol.  The parameter \textit{input-stream} is the
input stream from which the characters were read.  The parameter
\textit{token} is a string that contains all the characters of the
token.  The parameter \textit{position-package-marker-1} contains the
index into \textit{token} of the first package marker, or \texttt{nil}
if the token contains no package markers.  The parameter
\textit{position-package-marker-2} contains the index into
\textit{token} of the second package marker, or \texttt{nil} if the
token contains no package markers or only a single package marker.

The default method on this generic function calls
\texttt{interpret-symbol} (see below) with a symbol name string and a
package indicator.

\Defgeneric {interpret-symbol} {client input-stream package-indicator \\
  symbol-name internp}

This generic function is called by the default method on
\texttt{interpret-symbol-token} as well as the default
\texttt{\#:}~reader macro function to resolve a symbol name string and
a package indicator to a representation of the designated symbol.  The
parameter \textit{input-stream} is the input stream from which
\textit{package-indicator} and \textit{symbol-name} were read.  The
parameter \textit{package-indicator} is a either

\begin{itemize}
\item a string designating the package of that name
\item the keyword \texttt{:current} designating the current package
\item the keyword \texttt{:keyword} designating the keyword package
\item \texttt{nil} to indicate that an uninterned symbol should be
  created
\end{itemize}

The \textit{symbol-name} is the name of the desired symbol.

The default method uses \texttt{cl:find-package} (or
\texttt{cl:*package*} when \textit{package-indicator} is
\texttt{:current}) to resolve \textit{package-indicator} followed by
\texttt{cl:find-symbol} or \texttt{cl:intern}, depending on
\textit{internp}, to resolve \textit{symbol-name}.

A second method which is specialized on \textit{package-indicator}
being \texttt{nil} uses \texttt{cl:make-symbol} to create uninterned
symbols.

\Defgeneric {call-reader-macro} {client input-stream char readtable}

This generic function is called when the reader has determined that
some character is associated with a reader macro.  The parameter
\textit{char} has to be used in conjunction with the
\textit{readtable} parameter to obtain the macro function that is
associated with the macro character.  The parameter
\textit{input-stream} is the input stream from which the reader macro
function will read additional input to accomplish its task.

The default method on this generic function simply obtains the reader
macro function for \textit{char} from \textit{readtable} and calls it,
passing \textit{input-stream} and \textit{char} as arguments.  The
default method therefore does the same thing that the standard
\commonlisp{} reader does.

\Defgeneric {find-character} {client name}

This generic function is called by the default
\texttt{\#\textbackslash}~reader macro function to find a character by
name.  \textit{name} is the name that has been read converted to upper
case.  The function has to either return the character designated by
\textit{name} or \texttt{nil} if no such character exists.

\Defgeneric {make-structure-instance} {client name initargs}

This generic function is called by the default \texttt{\#S}~reader
macro function to construct structure instances.  \textit{name} is a
symbol naming the structure type of which an instance should be
constructed.  \textit{initargs} is a list the elements of which
alternate between symbols naming structure slots and values for those
slots.

There is no default method on this generic function since there is no
portable way to construct structure instances given only the name of
the structure type.

\Defgeneric {evaluate-expression} {client expression}

This generic function is called by the default \texttt{\#.}~reader
macro function to perform read-time evaluation.  \textit{expression}
is the expression that should be evaluated as it was returned by a
recursive \texttt{read} call and potentially influenced by
\textit{client}.  The function has to either return the result of
evaluating \textit{expression} or signal an error.

The default method on this generic function simply returns the result
of \texttt{(cl:eval expression)}.

\Defgeneric {check-feature-expression} {client feature-expression}

This generic function is called by the default \texttt{\#+}- and
\texttt{\#-}-reader macro functions to check the well-formedness of
\textit{feature-expression} which has been read from the input stream
before evaluating it.  For compound expressions, only the outermost
expression is checked regarding the atom in operator position and its
shape -- child expressions are not checked.  The function returns an
unspecified value if \textit{feature-expression} is well-formed and
signals an error otherwise.

The default method on this generic function accepts standard
\commonlisp{} feature expression, i.e.\ expressions recursively
composed of symbols, \texttt{:not}-expressions,
\texttt{:and}-expressions and \texttt{:or}-expressions.

\Defgeneric {evaluate-feature-expression} {client feature-expression}

This generic function is called by the default \texttt{\#+}- and
\texttt{\#-}-reader macro functions to evaluate
\textit{feature-expression} which has been read from the input stream.
The function returns either true or false if
\textit{feature-expression} is well-formed and signals an error
otherwise.

For compound feature expressions, the well-formedness of child
expressions is not checked immediately but lazily, just before the
child expression in question is evaluated in a subsequent
\texttt{evaluate-feature-expression} call.  This allows feature
expression like \texttt{\#+(and my-cl-implementation (special-feature
  a b))} to succeed when the \texttt{:my-cl-implementation} feature is
absent.

The default method on this generic function first calls
\texttt{check-feature-expression} to check the well-formedness of
\textit{feature-expression}.  It then evaluates
\textit{feature-expression} according to standard \commonlisp{}
semantics for feature expressions.

\Defgeneric {fixup} {client object seen-objects mapping}

This generic function is potentially called to apply
circularity-related changes to the object constructed by the reader
before it is returned to the caller.  \textit{object} is the object
that should be modified.  \textit{seen-objects} is a \texttt{eq}-hash
table used to track already processed objects (see below).
\textit{mapping} is a hash table of substitutions, mapping marker
objects to replacement objects.  A method specialized on a class,
instances of which consists of parts, should modify \textit{object} by
scanning its parts for marker objects, replacing found markers with
replacement object and recursively calling \texttt{fixup} for all
parts.  \texttt{fixup} is called for side effects -- its return value
is ignored.

Default methods specializing on the \textit{object} parameter for
\texttt{cons}, \texttt{array}, \texttt{standard-object} and
\texttt{hash-table} process instances of those classes in the obvious
way.

An unspecialized \texttt{:around} method queries and updates
\textit{seen-objects} to ensure that each object is processed exactly
once.

\Defgeneric {wrap-in-quote} {client material}

This generic function is called by the default \texttt{'}-reader macro
function to construct a quotation form in which \textit{material} is
the quoted material.

The default method on this generic function returns a result
equivalent to \texttt{(list 'common-lisp:quote material)}.

\Defgeneric {wrap-in-quasiquote} {client form}

This generic function is called by the default \texttt{`}-reader macro
function to construct a quasiquotation form in which \textit{form} is
the quasiquoted material.

The default method on this generic function returns a result
equivalent to \texttt{(list 'eclector.reader:quasiquote form)}.

\Defgeneric {wrap-in-unquote} {client form}

This generic function is called by the default \texttt{,}-reader macro
function to construct an unquote form in which \textit{form} is the
unquoted material.

The default method on this generic function returns a result
equivalent to \texttt{(list 'eclector.reader:unquote form)}.

\Defgeneric {wrap-in-unquote-splicing} {client form}

This generic function is called by the default \texttt{,@}-reader
macro function to construct a splicing unquote form in which
\textit{form} is the unquoted material.

The default method on this generic function returns a result
equivalent to \texttt{(list 'eclector.reader:unquote-splicing form)}.

\subsubsection{Readtable Initialization}
\label{sec:readtable-initialization}

The standard syntax types and macro character associations used by the
ordinary reader can be set up for any readtable object implementing
the readtable protocol \seesec{sec:readtable-features}  The following
functions are provided for this purpose:

\Defun {set-standard-syntax-types} (readtable)

This function sets the standard syntax types in \textit{readtable}
(See HyperSpec section 2.1.4.)

\Defun {set-standard-macro-characters} (readtable)

This function sets the standard macro characters in \textit{readtable}
(See HyperSpec section 2.4.)

\Defun {set-standard-dispatch-macro-characters} (readtable)

This function sets the standard dispatch macro characters, that is
sharpsign and its sub-characters, in \textit{readtable} (See HyperSpec
section 2.4.8.)

\Defun {set-standard-syntax-and-macros} (readtable)

This function sets the standard syntax types and macro characters in
\textit{readtable} by calling the above three functions.

\section{Readtable Features}
\label{sec:readtable-features}

In this section, symbols written without package marker are in the
\texttt{eclector.readtable} package
\seesec{sec:package-readtable-features}

TODO

\section{Parse result construction features}
\label{sec:parse-result-construction-features}

In this section, symbols written without package marker are in the
\texttt{eclector.parse-result} package
\seesec{sec:package-parse-result}

This package provides clients with a reader that behaves similarly to
\texttt{cl:read} but returns custom parse result objects controlled by
the client.  Some parse results correspond to things like symbols,
numbers and lists that \texttt{cl:read} would return, while others, if
the client chooses, represent comments and other kind of input that
\texttt{cl:read} would discard.  Furthermore, clients can associate
source location information with parse results.

Clients using this package must bind the special variable
\texttt{eclector.reader:*client*} around calls to \texttt{read} to an
instance for which methods on the generic functions described below
are applicable.  Suitable client classes can be constructed by using
\texttt{parse-result-client} as a superclass and at least defining a
method on the generic function \texttt{make-expression-result}.

\Defun {read} {client \optional (input-stream \texttt{*standard-input*})\\
  (eof-error-p \texttt{t})
  (eof-value \texttt{nil})}

This function is the main entry point for this variant of the reader.
It is in many ways similar to the standard \commonlisp{} function
\texttt{read}.  The differences are:

\begin{itemize}
\item A client instance must be supplied as the first argument.
\item The first return value, unless \textit{eof-value} is returned,
  is an arbitrary parse result object created by the client, not
  generally the read object.
\item The second return value, unless \textit{eof-value} is returned,
  is a list of ``orphan'' results.  These results are return values of
  \texttt{make-skipped-input-result} and arise when skipping input at
  the toplevel such as comments which are not lexically contained in
  lists: \texttt{\#|orphan|\# (\#|not orphan|\#)}.
\item The function does not accept a \textit{recursive} parameter
  since it sets up a dynamic environment in which calls to
  \texttt{eclector.reader:read} behave suitably.
\end{itemize}

\Defclass {parse-result-client}

This class should generally be used as a superclass for client classes
using this package.

\Defgeneric {source-position} {client stream}

This generic function is called in order to determine the current
position in \textit{stream}.  The default method calls
\texttt{cl:file-position}.

\Defgeneric {make-source-range} {client start end}

This generic function is called in order to turn the source positions
\textit{start} and \textit{end} into a range representation suitable
for \textit{client}.  The returned representation designates the range
of input characters from and including the character at position
\textit{start} to but not including the character at position
\textit{end}.  The default method returns \texttt{(cons start end)}.

\Defgeneric {make-expression-result} {client result children source}

This generic function is called in order to construct a parse result
object.  The value of the \textit{result} parameter is the raw object
read.  The value of the \textit{children} parameter is a list of
already constructed parse result objects representing objects read by
recursive \texttt{read} calls.  The value of the \textit{source}
parameter is a source range, as returned by \texttt{make-source-range}
and \texttt{source-position} delimiting the range of characters from
which \textit{result} has been read.

This generic function does not have a default method since the purpose
of the package is the construction of \emph{custom} parse results.
Thus, a client must define a method on this generic function.

\Defgeneric {make-skipped-input-result} {client stream reason source}

This generic function is called after the reader skipped over a range
of characters in \textit{stream}.  It returns either \texttt{nil} if
the skipped input should not be represented or a client-specific
representation of the skipped input.  The value of the \textit{source}
parameter designates the skipped range using a source range
representation obtained via \texttt{make-source-range} and
\texttt{source-position}.

Reasons for skipping input include comments, the \texttt{\#+} and
\texttt{\#-} reader macros and \texttt{*read-suppress*}.  The
aforementioned reasons are reflected by the value of the
\textit{reason} parameter as follows:

\begin{tabular}{ll}
  Input                                          & Value of the \textit{reason} parameter\\
  \hline
  Comment starting with \texttt{;}               & \texttt{(:line-comment~.~1)}\\
  Comment starting with \texttt{;;}              & \texttt{(:line-comment~.~2)}\\
  Comment starting with $n$ \texttt{;}           & \texttt{(:line-comment~.~$n$)}\\
  Comment delimited by \texttt{\#|} \texttt{|\#} & \texttt{:block-comment}\\
  \texttt{\#+\textit{false-feature-expression}}  & \texttt{:reader-macro}\\
  \texttt{\#-\textit{true-feature-expression}}   & \texttt{:reader-macro}\\
  \texttt{*read-suppress*} is true               & \texttt{*read-suppress*}
\end{tabular}

The default method returns \texttt{nil}, that is the skipped input is
not represented as a parse result.

\section{CST reader features}
\label{sec:cst-reader-features}

In this section, symbols written without package marker are in the
\texttt{eclector.concrete-syntax-tree} package
\seesec{sec:package-cst-features}

\Defun {cst-read} {\optional (input-stream \texttt{*standard-input*})\\
  (eof-error-p \texttt{t})
  (eof-value \texttt{nil})}

This function is the main entry point for the CST reader.  It is
mostly compatible with the standard \commonlisp{} function
\texttt{read}.  The differences are:

\begin{itemize}
\item The return value, unless \textit{eof-value} is returned, is an
  instance of a subclass of \texttt{concrete-syntax-tree:cst}.
\item The function does not accept a \textit{recursive} parameter
  since it sets up a dynamic environment in which calls to
  \texttt{eclector.reader:read} behave suitably.
\end{itemize}
