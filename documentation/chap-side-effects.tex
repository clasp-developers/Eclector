\chapter{Side effects}
\label{chap:side-effects}

This chapter describes potential side effects of calling
\texttt{eclector.reader:read},
\texttt{eclector.reader:read-preserving-whitespace} or
\texttt{eclector.reader:read-from-string} for different kinds of
clients.

\section{Potential side effects for the default client}
\label{sec:side-effects-default-client}

The following destructive modifications are considered uninteresting
and ignored in the remainder of this section:

\begin{itemize}
\item Changes to the state of streams passed to the functions
  mentioned above.
\item Changes to objects within expressions currently being read.
\end{itemize}

Furthermore, the remainder of this section is written under the
following assumptions:

\begin{itemize}
\item The stream object passed to \texttt{eclector.reader:read} does
  not cause additional side effects on its own.
\item The variable \texttt{eclector.reader:*client*} is bound to an
  object for which there are no custom applicable methods on generic
  functions belonging to protocols provided by \sysname{} that
  introduce additional side effects.
\item The variable \texttt{eclector.readtable:*readtabe*} is bound to
  an object for which
  \begin{itemize}
  \item there are no custom applicable methods on generic functions
    belonging to protocols provided by \sysname{} that introduce
    additional side effects
  \item no non-default macro functions have been installed
  \end{itemize}
\end{itemize}

If any of the above assumptions does not hold, ``all bets are off'' in
the sense that arbitrary side effects other than the ones described
below are possible.  For notes regarding non-default clients,
see \refsec{sec:side-effects-non-default-clients}.

\subsection{Symbols and packages}
\label{sec:default-symbols-and-packages}

The default method on the generic function
\texttt{eclector.reader:interpret-symbol} may create and intern
symbols, thereby modifying the package system.

\subsection{Read-time evaluation}
\label{sec:default-read-time-evaluation}

The default method on the generic function
\texttt{eclector.reader:evaluate-expression} uses \texttt{cl:eval} to
evaluate arbitrary expressions, potentially causing side effects.
With the default readtable, the generic function is only called by the
macro function of the \texttt{\#.} reader macro.

\subsection{Standard reader macros}
\label{sec:default-standard-reader-macros}

The default method on the generic function
\texttt{eclector.reader:call-reader-macro} can cause side effects by
calling macro functions that cause side effects.  The following
standard reader macros potentially cause side-effects:

\begin{itemize}
\item \texttt{\#.} as described in \refsec{sec:default-read-time-evaluation}.
\end{itemize}

\section{Potential side effects for non-default clients}
\label{sec:side-effects-non-default-clients}

\subsection{Symbols and packages}
\label{sec:non-default-symbols-and-packages}

In addition to the potential side effects described in
\refsec{sec:default-symbols-and-packages}, strings passed as the third
argument of \texttt{eclector.reader:interpret-token} are potentially
destructively modified during conversion to the current readtable
case.

\subsection{Read-time evaluation}
\label{sec:non-default-read-time-evaluation}

The same considerations as in
\refsec{sec:default-read-time-evaluation} apply.

\subsection{Structure instance creation}
\label{sec:non-default-structure-instance-creation}

Clients defining methods on
\texttt{eclector.reader:make-structure-instance} which implement the
standard behavior of calling the default constructor (if any) of the
named structure should consider side effects caused by slot initforms
of the structure.  The following example illustrates this problem:
\begin{Verbatim}[frame=single]
  (defvar *counter* 0)
  (defstruct foo (bar (incf *counter*)))
  #S(foo)
  *counter* ; => 1
  #S(foo)
  *counter* ; => 2
\end{Verbatim}

\subsection{Circular structure}
\label{sec:circular-structure}

The \texttt{fixup} generic function potentially modifies its second
argument destructively.  Clients that define methods on
\texttt{eclector.reader:make-structure-instance} should be aware of
this potential modification in cases like \texttt{\#1=\#S(foo :bar
  \#1\#)}.  Similar considerations apply for other ways of
constructing compound objects such as \texttt{\#1=(t . \#1\#)}.

\subsection{Standard reader macros}
\label{sec:non-default-standard-reader-macros}

The following standard reader macros could cause or be affected by
side effects when combined with a non-standard client:

\begin{itemize}
\item \texttt{\#.} as described in
  \refsec{sec:default-read-time-evaluation}.
\item \texttt{\#S} as described in
  \refsec{sec:non-default-structure-instance-creation}.
\item \texttt{(}, \texttt{\#(} and \texttt{\#S} as described in
  \refsec{sec:circular-structure}.
\item The \texttt{,.} (i.e. destructively splicing) variant of the
  \texttt{,} reader macro does not currently destructively modify the
  surrounding object, but clients should not rely on this fact.  This
  consideration applies to clients that install non-standard macro
  functions for the \texttt{(} and \texttt{\#(} reader macros.
\end{itemize}
